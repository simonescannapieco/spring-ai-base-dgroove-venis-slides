\section{\faWrench\ Implicit e zero-shot prompting} % (fold)
\label{sec:spring-ai-gemini-implicit-zero-shot-prompting}
%
\begin{frame}[t,fragile] \frametitle{Progetto Spring AI}
    \framesubtitle{Descrizione - I}
    {\small
        \begin{itemize}[leftmargin=10pt,align=right]
            \onslide<1->\item[\alertedcircled{1}] \textit{Implicit prompting}
            \begin{itemize}[leftmargin=10pt,align=right]
                \item[\alert{\faArrowCircleRight}] Strategia: atto di fede nei confronti del LLM
                \begin{itemize}[leftmargin=10pt,align=right]
                    \item[\alert{\faArrowCircleRight}] Utente fornisce solo l'\textit{input} su cui lavorare
                    \item[\alert{\faArrowCircleRight}] \textit{Output} del LLM determinato dalle sue \alert{capacità emergenti} 
                    \item[\alert{\faExclamationTriangle}] LLM determina la \textit{task} dal contenuto dell'\textit{input} e dalla sua formattazione
                    \item[\alert{\faExclamationTriangle}] Preferire formato strutturato di \textit{markdown} nel definire l'\textit{input} (rendere il processo di \alert{cattura dei \textit{pattern}} più semplice)
                \end{itemize}
                \onslide<2->\item[\alert{\faArrowCircleRight}] Vantaggi
                \begin{itemize}[leftmargin=10pt,align=right]
                    \item[\alert{\faArrowCircleRight}] Semplice da utilizzare
                \end{itemize}
                \onslide<3->\item[\alert{\faArrowCircleRight}] Svantaggi
                \begin{itemize}[leftmargin=10pt,align=right]
                    \item[\alert{\faArrowCircleRight}] Visto come esercizio di stile
                    \item[\alert{\faArrowCircleRight}] Molto limitato a \textit{task} semplici (focalizzate sul puro significato dei \textit{token})
                \end{itemize}               
            \end{itemize}
        \end{itemize}
    }
\end{frame}
%
\begin{frame}[t,fragile] \frametitle{Progetto Spring AI}
    \framesubtitle{Descrizione - II}
    {\small
        \begin{itemize}[leftmargin=10pt,align=right]
            \onslide<1->\item[\alertedcircled{2}] \textit{Zero-shot prompting}
            \begin{itemize}[leftmargin=10pt,align=right]
                \item[\alert{\faArrowCircleRight}] Strategia: \textit{implicit prompting} + direttive
                \begin{itemize}[leftmargin=10pt,align=right]
                    \item[\alert{\faArrowCircleRight}] Utente \alert{non} fornisce esempi
                \end{itemize}
                \onslide<2->\item[\alert{\faArrowCircleRight}] Vantaggi
                \begin{itemize}[leftmargin=10pt,align=right]
                    \item[\alert{\faArrowCircleRight}] Semplice da utilizzare
                \end{itemize}
                \onslide<3->\item[\alert{\faArrowCircleRight}] Svantaggi
                \begin{itemize}[leftmargin=10pt,align=right]
                    \item[\alert{\faArrowCircleRight}] Molto limitato a \textit{task} semplici
                    \item[\alert{\faArrowCircleRight}] Perché non utilizzare esempi\ldots?
                \end{itemize}               
            \end{itemize}
        \end{itemize}
    }
\end{frame}
%
\begin{frame}[t,fragile] \frametitle{Progetto Spring AI}
    \framesubtitle{Passaggi}
    {\small
        \begin{itemize}[leftmargin=10pt,align=right]
            \item[\alert{\faArrowCircleRight}] Sistema di \alert{traduzione automatica} da italiano a multilingua
            \begin{itemize}[leftmargin=10pt,align=right]
                \onslide<2->\item[\alertedcircled{1}] Creazione modello \texttt{TranslationRequest.java} per richiesta traduzione lemma
                \onslide<3->\item[\alertedcircled{2}] Creazione \textit{prompt template} per traduzione da italiano a multilingua 
                \onslide<4->\item[\alertedcircled{3}] Modifiche ad interfaccia ed implementazione del servizio Gemini
                \onslide<5->\item[\alertedcircled{4}] Modifica del controllore MVC per servizio Gemini
                \onslide<6->\item[\alertedcircled{5}] \textit{Test} delle funzionalità con Postman/Insomnia           
            \end{itemize}
            \onslide<7->\item[\alert{\faArrowCircleRight}] Sistema di \alert{\textit{sentiment analysis}} per artefatto
            \begin{itemize}[leftmargin=10pt,align=right]
                \onslide<8->\item[\alertedcircled{1}] Creazione enumeratori \texttt{ArtifactType.java} e \texttt{Sentiment.java}
                \onslide<9->\item[\alertedcircled{2}] Creazione modello di artefatto \texttt{Artifact.java}
                \onslide<10->\item[\alertedcircled{3}] Creazione modello richiesta per artefatto \texttt{ArtifactRequest.java}
                \onslide<11->\item[\alertedcircled{4}] Creazione \textit{prompt template} per \textit{sentiment analysis} di tipo \textit{0-shot} per artefatti
                \onslide<12->\item[\alertedcircled{5}] Modifiche ad interfaccia ed implementazione del servizio Gemini
                \onslide<13->\item[\alertedcircled{6}] Modifica del controllore MVC per servizio Gemini
                \onslide<14->\item[\alertedcircled{7}] \textit{Test} delle funzionalità con Postman/Insomnia
            \end{itemize}
        \end{itemize}
    }
\end{frame}
%
\section{\faWrench\ Traduzione multilingua} % (fold)
\label{sec:spring-ai-gemini-zero-implicit-prompting-translation}
%
\begin{frame}[t,fragile] \frametitle{Progetto Spring AI}
    \framesubtitle{Traduzione}
        \begin{block}{Modello per richiesta traduzione}
			{\tiny\input{code/TranslationRequest.java}}
    	\end{block}
\end{frame}
%
\begin{frame}[t,fragile] \frametitle{Progetto Spring AI}
    \framesubtitle{Prompt per traduzione multilingua}
        \begin{itemize}[leftmargin=10pt,align=right]
		    \item[\alert{\faExclamationTriangle}] Il \textit{path} \texttt{resources/templates} viene creato automaticamente da Spring AI in fase di creazione del progetto
        \end{itemize}
        \begin{block}{\textit{File} \texttt{get-lemma-translation-prompt.st}}
			{\scriptsize\input{code/get-lemma-translation-prompt.st}}
    	\end{block}
\end{frame}
%
\begin{frame}[t,fragile] \frametitle{Progetto Spring AI}
    \framesubtitle{Servizio Gemini}
        \begin{block}{Interfaccia servizio Gemini}
{\tiny\input{code/GeminiFromClientService-2.java}}
    \end{block}
\end{frame}
%
\begin{frame}[t,fragile] \frametitle{Progetto Spring AI}
    \framesubtitle{Servizio Gemini}
		\vspace*{-.7cm}
        \begin{block}{Implementazione servizio Gemini - I}
            {\tiny\input{code/GeminiFromClientServiceImpl-3.java}}
    \end{block}
\end{frame}
%
\begin{frame}[t,fragile] \frametitle{Progetto Spring AI}
    \framesubtitle{Servizio Gemini}
    	\vspace*{-.7cm}
        \begin{block}{Implementazione servizio Gemini - II}
            {\tiny\input{code/GeminiFromClientServiceImpl-4.java}}
    \end{block}
\end{frame}
%
\begin{frame}[t,fragile] \frametitle{Progetto Spring AI}
    \framesubtitle{MVC del servizio Gemini}
    	\vspace*{-.7cm}
        \begin{block}{Implementazione controllore REST}
			{\tiny\input{code/QuestionFromClientController-2.java}}
    	\end{block}
\end{frame}
%
\begin{frame}[t,fragile] \frametitle{Progetto Spring AI}
    \framesubtitle{\ldots e ora a voi}
    {\small
        \begin{itemize}[leftmargin=10pt,align=right]
            \onslide<1->\item[\alert{\faArrowCircleRight}] Sistema di traduzione con \alert{lingua parametrizzata}
            \begin{itemize}[leftmargin=10pt,align=right]
                \item[\alert{\faArrowCircleRight}] \texttt{language} come parametro aggiuntivo del modello
                \item[\alert{\faArrowCircleRight}] \textit{Output} atteso con \texttt{\{''lemma'':''aid''}, \texttt{''language'':''ENG''\}} è \texttt{ITA:aiuto,ESP:aiudo,\ldots}  
            \end{itemize}
            \onslide<2->\item[\alert{\faArrowCircleRight}] Sistema di \alert{sinonimi e contrari} per lingua italiana
        \end{itemize}
    }
\end{frame}
%
\section{\faWrench\ Sentiment analysis} % (fold)
\label{sec:spring-ai-gemini-zero-shot-prompting-sentiment-analysis}
%
\begin{frame}[t,fragile] \frametitle{Progetto Spring AI}
    \framesubtitle{Tipo di artefatto}
        \begin{block}{Enumeratore per tipo di artefatto}
			{\tiny\input{code/ArtifactType.java}}
    	\end{block}
\end{frame}
%
\begin{frame}[t,fragile] \frametitle{Progetto Spring AI}
    \framesubtitle{\textit{Sentiment}}
        \begin{block}{Enumeratore per \textit{sentiment}}
			{\tiny\input{code/Sentiment.java}}
    	\end{block}
\end{frame}
%
\begin{frame}[t,fragile] \frametitle{Progetto Spring AI}
    \framesubtitle{Artefatto}
        \begin{block}{Modello di artefatto}
			{\tiny\input{code/Artifact.java}}
    	\end{block}
\end{frame}
%
\begin{frame}[t,fragile] \frametitle{Progetto Spring AI}
    \framesubtitle{Artefatto}
        \begin{block}{Modello di richiesta per artefatto}
			{\tiny\input{code/ArtifactRequest.java}}
    	\end{block}
\end{frame}
%
\begin{frame}[t,fragile] \frametitle{Progetto Spring AI}
    \framesubtitle{Prompt per sentiment analysis}
        \begin{itemize}[leftmargin=10pt,align=right]
		    \item[\alert{\faExclamationTriangle}] Il \textit{path} \texttt{resources/templates} viene creato automaticamente da Spring AI in fase di creazione del progetto
        \end{itemize}
        \begin{block}{\textit{File} \texttt{get-artifact-sentiment-prompt.st}}
			{\scriptsize\input{code/get-artifact-sentiment-prompt.st}}
    	\end{block}
\end{frame}
%
\begin{frame}[t,fragile] \frametitle{Progetto Spring AI}
    \framesubtitle{Servizio Gemini}
        \begin{block}{Interfaccia servizio Gemini}
{\tiny\input{code/GeminiFromClientService.java}}
    \end{block}
\end{frame}
%
\begin{frame}[t,fragile] \frametitle{Progetto Spring AI}
    \framesubtitle{Servizio Gemini}
		\vspace*{-.7cm}
        \begin{block}{Implementazione servizio Gemini - I}
            {\tiny\input{code/GeminiFromClientServiceImpl.java}}
    \end{block}
\end{frame}
%
\begin{frame}[t,fragile] \frametitle{Progetto Spring AI}
    \framesubtitle{Servizio Gemini}
    	\vspace*{-.7cm}
        \begin{block}{Implementazione servizio Gemini - II}
            {\tiny\input{code/GeminiFromClientServiceImpl-2.java}}
    \end{block}
\end{frame}
%
\begin{frame}[t,fragile] \frametitle{Progetto Spring AI}
    \framesubtitle{MVC del servizio Gemini}
    	\vspace*{-.7cm}
        \begin{block}{Implementazione controllore REST}
			{\tiny\input{code/QuestionFromClientController.java}}
    	\end{block}
\end{frame}
%
\begin{frame}[fragile] \frametitle{Codice}
    \framesubtitle{Branch di riferimento}
	\begin{center}
		{\scriptsize \href{https://github.com/simonescannapieco/spring-ai-base-dgroove-venis-code.git}{\texttt{https://github.com/simonescannapieco/spring-ai-base-dgroove-venis-code.git}}}\\
		\textit{Branch:} \alert{\texttt{13-spring-ai-gemini-implicit-zero-shot-prompting}}
	\end{center}
\end{frame}