\section{\faWrench\ One-shot/Few-shot prompting} % (fold)
\label{sec:spring-ai-gemini-few-shot-prompting}
%
\begin{frame}[t,fragile] \frametitle{Progetto Spring AI}
    \framesubtitle{Descrizione}
    {\small
        \begin{itemize}[leftmargin=10pt,align=right]
            \onslide<1->\item[\alertedcircled{1}] \textit{One-shot/Few-shot prompting}
            \begin{itemize}[leftmargin=10pt,align=right]
                \item[\alert{\faArrowCircleRight}] Strategia: fornire esempi di corretto comportamento al modello in relazione alla \textit{task} da eseguire
                \begin{itemize}[leftmargin=10pt,align=right]
                    \item[\alert{\faArrowCircleRight}] Aggiunge un \textit{layer} informativo assistivo alle direttive
                    \item[\alert{\faExclamationTriangle}] \textit{Few-shot} da preferire all'aumentare della complessità della \textit{task}
                    \item[\alert{\faExclamationTriangle}] Preferire formato strutturato di \textit{markdown} nel definire gli esempi (rendere il processo di \alert{cattura dei \textit{pattern}} più semplice)
                \end{itemize}
                \onslide<2->\item[\alert{\faArrowCircleRight}] Vantaggi
                \begin{itemize}[leftmargin=10pt,align=right]
                    \item[\alert{\faArrowCircleRight}] \alert{Per \textit{one-shot}}, risultati efficaci con \textit{task} piuttosto complesse (proprietà emergenti)
                    \item[\alert{\faArrowCircleRight}] \alert{Per \textit{few-shot}}, risultati efficaci con numero di esempi relativamente contenuto (da 3 a 5)
                \end{itemize}
                \onslide<3->\item[\alert{\faArrowCircleRight}] Svantaggi
                \begin{itemize}[leftmargin=10pt,align=right]
                    \item[\alert{\faArrowCircleRight}] Farraginoso e \textit{time-consuming} lato utente
                    \item[\alert{\faArrowCircleRight}] Limitazioni di \textit{context window} con LLM ``meno famosi''
                \end{itemize}               
            \end{itemize}
        \end{itemize}
    }
\end{frame}
%
\begin{frame}[t,fragile] \frametitle{Progetto Spring AI}
    \framesubtitle{Applicazione e passaggi}
    {\small
        \begin{itemize}[leftmargin=10pt,align=right]
            \onslide<1->\item[\alert{\faArrowCircleRight}] Sistema di \alert{estrazione entità denominate} (NER)
            \begin{itemize}[leftmargin=10pt,align=right]
                \onslide<2->\item[\alertedcircled{1}] Creazione \textit{string template} con esempi di estrazione NER da articolo di giornale
                \onslide<3->\item[\alertedcircled{2}] Modifiche ad interfaccia ed implementazione del servizio Gemini
                \onslide<4->\item[\alertedcircled{3}] Modifica del controllore MVC per servizio Gemini
                \onslide<5->\item[\alertedcircled{4}] \textit{Test} delle funzionalità con Postman/Insomnia 
            \end{itemize}
        \end{itemize}
    }
\end{frame}
%
\begin{frame}[t,fragile] \frametitle{Progetto Spring AI}
    \framesubtitle{Prompt per sentiment analysis}
        \vspace*{-.7cm}
        \begin{block}{\textit{File} \texttt{get-ner-yaml-prompt.st}}
			{\Tiny\input{code/get-ner-yaml-prompt.st}}
    	\end{block}
\end{frame}
%
\begin{frame}[t,fragile] \frametitle{Progetto Spring AI}
    \framesubtitle{Servizio Gemini}
        \begin{block}{Interfaccia servizio Gemini}
{\tiny\input{code/GeminiFromClientService.java}}
    \end{block}
\end{frame}
%
\begin{frame}[t,fragile] \frametitle{Progetto Spring AI}
    \framesubtitle{Servizio Gemini}
		\vspace*{-.7cm}
        \begin{block}{Implementazione servizio Gemini - I}
            {\tiny\input{code/GeminiFromClientServiceImpl.java}}
    \end{block}
\end{frame}
%
\begin{frame}[t,fragile] \frametitle{Progetto Spring AI}
    \framesubtitle{Servizio Gemini}
        \vspace*{-.7cm}
        \begin{block}{Implementazione servizio Gemini - II}
            {\tiny\input{code/GeminiFromClientServiceImpl-2.java}}
    \end{block}
\end{frame}
%
\begin{frame}[t,fragile] \frametitle{Progetto Spring AI}
    \framesubtitle{MVC del servizio Gemini}
    	\vspace*{-.7cm}
        \begin{block}{Implementazione controllore REST}
			{\tiny\input{code/QuestionFromClientController.java}}
    	\end{block}
\end{frame}
%
\begin{frame}[t,fragile] \frametitle{Progetto Spring AI}
    \framesubtitle{\ldots e ora a voi}
    {\small
        \begin{itemize}[leftmargin=10pt,align=right]
            \onslide<1->\item[\alert{\faArrowCircleRight}] Sistema di NER da \alert{\textit{one-shot} a \textit{few-shot}}
            \begin{itemize}[leftmargin=10pt,align=right]
                \item[\alert{\faArrowCircleRight}] Aggiungere da 3 a 5 articoli come esempi
                \item[\alert{\faArrowCircleRight}] Introdurre concetti NER come \texttt{data}, \texttt{organizzazione}, \texttt{orario}, \texttt{valuta}, \texttt{percentuale}, \texttt{quantità} 
                \item[\alert{\faArrowCircleRight}] Imporre un formato standardizzato (es. orari \texttt{HH:MM}, data \texttt{DD/MM/YYYY})  
            \end{itemize}
        \end{itemize}
    }
\end{frame}
%
\begin{frame}[fragile] \frametitle{Codice}
    \framesubtitle{Branch di riferimento}
	\begin{center}
		{\scriptsize \href{https://github.com/simonescannapieco/spring-ai-base-dgroove-venis-code.git}{\texttt{https://github.com/simonescannapieco/spring-ai-base-dgroove-venis-code.git}}}\\
		\textit{Branch:} \alert{\texttt{14-spring-ai-gemini-few-shot-prompting}}
	\end{center}
\end{frame}