\section{Spring AI: Introduzione} % (fold)
\label{sec:spring_ai_intro}
%
\begin{frame}[t,fragile] \frametitle{Spring AI}
	{\scriptsize
		\onslide<1->
            \framesubtitle{L'evoluzione dell'ecosistema Spring}
            \vspace*{-15pt}
            \begin{minipage}[t]{\textwidth}
             	%\begin{figure}[ht]
                %    \centering
                %    \includegraphics[width=0.8\textwidth]{img/spring-ai-logo.png}
                %\end{figure}
            \end{minipage}
            \\\vspace*{3pt}
	    	\begin{minipage}[t]{\textwidth}
				\begin{minipage}[t]{0.6\textwidth}
	    			\begin{itemize}[leftmargin=10pt,align=right]
						\onslide<2->\item[\alert{\faArrowCircleRight}] Framework per \alert{integrare l'Intelligenza Artificiale} nelle applicazioni Spring Boot
						\onslide<3->\item[\alert{\faArrowCircleRight}] \alert{Astrazione unificata} per diversi provider AI (OpenAI, Azure OpenAI, AWS Bedrock, Google Vertex AI)
						\onslide<4->\item[\alert{\faArrowCircleRight}] Integrazione \alert{nativa} con l'ecosistema Spring esistente
						\onslide<5->\item[\alert{\faArrowCircleRight}] Supporto per \alert{Vector Stores}, \alert{Embedding} e \alert{Retrieval-Augmented Generation (RAG)}
					\end{itemize}
            	\end{minipage}
            	%
				\onslide<1->
            	\begin{minipage}[t]{0.4\textwidth}
                	%\centering
                	%\begin{figure}[ht]
                    %	\includegraphics[width=.8\textwidth]{img/spring-ecosystem.png}
                    %	{\tiny\\Ecosistema Spring\\\vspace*{-1pt}\textit{\textcopyright Spring Team}}
                	%\end{figure}
            	\end{minipage}
	    	\end{minipage}
	}
\end{frame}
%
\begin{frame}[t,fragile] \frametitle{Spring AI}
	{\scriptsize
		\onslide<1->
            \framesubtitle{Perché Spring AI?}
            \vspace*{-10pt}
	    	\begin{minipage}[t]{\textwidth}
				\begin{minipage}[t]{0.6\textwidth}
	    			\begin{itemize}[leftmargin=10pt,align=right]
						\onslide<2->\item[\alert{\faArrowCircleRight}] \alert{Problema:} ogni provider AI ha API e modelli diversi
						\onslide<3->\item[\alert{\faArrowCircleRight}] \alert{Soluzione Spring AI:} interfaccia unificata e portabile
						\onslide<4->\item[\alert{\faArrowCircleRight}] \alert{Principi Spring:}
						\begin{itemize}[leftmargin=10pt,align=right]
							\item[\alert{\faArrowCircleRight}] Dependency Injection
							\item[\alert{\faArrowCircleRight}] Auto-configuration
							\item[\alert{\faArrowCircleRight}] Testabilità
							\item[\alert{\faArrowCircleRight}] Observability
						\end{itemize}
						\onslide<5->\item[\alert{\faArrowCircleRight}] \alert{Integrazione} con Spring Boot, Spring Security, Spring Data
					\end{itemize}
            	\end{minipage}
            	%
				\onslide<1->
            	\begin{minipage}[t]{0.4\textwidth}
                	%\centering
                	%\begin{figure}[ht]
                    % 	\includegraphics[width=.8\textwidth]{img/ai-providers-chaos.png}
                    %	{\tiny\\Frammentazione dei provider AI\\\vspace*{-1pt}\textit{\textcopyright Tech Comparison}}
                	%\end{figure}
            	\end{minipage}
	    	\end{minipage}
	}
\end{frame}
%
\section{Architettura di Spring AI} % (fold)
\label{sec:spring_ai_architecture}
%
\begin{frame}[t,fragile] \frametitle{Architettura Spring AI}
    \framesubtitle{Componenti principali}
    \vspace*{-10pt}
    %\begin{figure}[ht]
    %    \centering
    %    \includegraphics[width=\textwidth]{img/spring-ai-architecture.png}
    %\end{figure}
    %\begin{flushright}
    %    \vspace*{-10pt}
    %    {\tiny\textit{\textcopyright Spring AI Documentation}}
    %\end{flushright}
\end{frame}
%
%
\begin{frame}[t,fragile] \frametitle{Chat Client}
	{\small
		\framesubtitle{Il cuore dell'interazione AI}
		\begin{itemize}[leftmargin=10pt,align=right]
			\onslide<2->\item[\alert{\faArrowCircleRight}] \alert{ChatClient:} interfaccia principale per conversazioni AI
			\onslide<3->\item[\alert{\faArrowCircleRight}] \alert{Fluent API} per costruire richieste complesse
			\onslide<4->\item[\alert{\faArrowCircleRight}] Supporto per \alert{streaming}, \alert{function calling} e \alert{multimodal}
		\end{itemize}
		\vspace*{.3cm}
		\onslide<1->
\begin{minted}[bgcolor=vscode-dark-bg, bgcolorpadding=0.5em]{java}
@RestController
public class ChatController {
    
    private final ChatClient chatClient;
    
    public ChatController(ChatClient.Builder chatClientBuilder) {
        this.chatClient = chatClientBuilder.build();
    }
    
    @PostMapping("/ai/chat")
    public String chat(@RequestBody String userInput) {
        return chatClient.prompt()
                .user(userInput)
                .call()
                .content();
    }
}
\end{minted}
	}
\end{frame}
%
\begin{frame}[t,fragile] \frametitle{Vector Stores}
	{\small
		\framesubtitle{Gestione della memoria semantica}
		\begin{itemize}[leftmargin=10pt,align=right]
			\onslide<2->\item[\alert{\faArrowCircleRight}] \alert{Vector Store:} database ottimizzati per similarity search
			\onslide<3->\item[\alert{\faArrowCircleRight}] Supporto per \alert{Redis}, \alert{Chroma}, \alert{Weaviate}, \alert{Pinecone}
			\onslide<4->\item[\alert{\faArrowCircleRight}] Integrazione con \alert{Spring Data} patterns
		\end{itemize}
		\vspace*{.3cm}
		\onslide<1->
\begin{minted}[bgcolor=vscode-dark-bg, bgcolorpadding=0.5em]{java}
@Configuration
public class VectorStoreConfig {
    
    @Bean
    public VectorStore vectorStore(EmbeddingClient embeddingClient) {
        return new SimpleVectorStore(embeddingClient);
    }
    
    @Bean 
    public DocumentRetriever documentRetriever(VectorStore vectorStore) {
        return VectorStoreDocumentRetriever.builder()
                .vectorStore(vectorStore)
                .topK(5)
                .similarityThreshold(0.7)
                .build();
    }
}
\end{minted}
	}
\end{frame}
%
\begin{frame}[t,fragile] \frametitle{Embedding Models}
	{\small
		\framesubtitle{Trasformazione testo in vettori}
		\begin{itemize}[leftmargin=10pt,align=right]
			\onslide<2->\item[\alert{\faArrowCircleRight}] \alert{EmbeddingClient:} interfaccia per modelli di embedding
			\onslide<3->\item[\alert{\faArrowCircleRight}] Trasforma testo in \alert{rappresentazioni vettoriali} per similarity search
			\onslide<4->\item[\alert{\faArrowCircleRight}] Supporto per \alert{OpenAI}, \alert{Ollama}, \alert{Azure OpenAI}
		\end{itemize}
		\vspace*{.3cm}
		\onslide<1>
\begin{minted}[bgcolor=vscode-dark-bg, bgcolorpadding=0.5em]{java}
@Service
public class DocumentService {
    
    private final EmbeddingClient embeddingClient;
    private final VectorStore vectorStore;
    
    public void addDocument(String content, Map<String, Object> metadata) {
        // Crea embedding del contenuto
        List<Float> embedding = embeddingClient.embed(content);
        
        // Salva nel vector store
        Document document = new Document(content, metadata);
        document.setEmbedding(embedding);
        
        vectorStore.add(List.of(document));
    }
}
\end{minted}
	}
\end{frame}
%
\section{Vantaggi di Spring AI} % (fold)
\label{sec:spring_ai_advantages}
%
\begin{frame}[t,fragile] \frametitle{Vantaggi di Spring AI}
	{\small
		\framesubtitle{Perché scegliere Spring AI}
		\begin{itemize}[leftmargin=10pt,align=right]
			\onslide<1->\item[\alert{\faArrowCircleRight}] \alert{Portabilità tra provider:} cambio di provider AI senza modificare codice business
			\onslide<2->\item[\alert{\faArrowCircleRight}] \alert{Integrazione Spring nativa:} sfrutta tutte le funzionalità dell'ecosistema
			\onslide<3->\item[\alert{\faArrowCircleRight}] \alert{Auto-configuration:} configurazione automatica basata su dependencies
			\onslide<4->\item[\alert{\faArrowCircleRight}] \alert{Testabilità migliorata:} mock e test integration semplificati
			\onslide<5->\item[\alert{\faArrowCircleRight}] \alert{Observability:} metriche e monitoring integrati
			\onslide<6->\item[\alert{\faArrowCircleRight}] \alert{Gestione errori:} retry policies e fallback automatici
		\end{itemize}
	}
\end{frame}
%
\begin{frame}[t,fragile] \frametitle{Portabilità tra Provider}
	{\small
		\framesubtitle{Un codice, molti provider}
		\onslide<1->
\begin{minted}[bgcolor=vscode-dark-bg, bgcolorpadding=0.5em]{java}
// Configurazione per OpenAI
@Configuration
public class OpenAIConfig {
    @Bean
    public ChatClient chatClient() {
        return ChatClient.builder()
                .chatModel(new OpenAiChatModel(apiKey))
                .build();
    }
}

// Configurazione per Azure OpenAI  
@Configuration
public class AzureConfig {
    @Bean  
    public ChatClient chatClient() {
        return ChatClient.builder()
                .chatModel(new AzureOpenAiChatModel(endpoint, apiKey))
                .build();
    }
}
\end{minted}
	\only<2>{
		\begin{center}
			\alert{Il business logic rimane identico!}
		\end{center}
		}
	}
\end{frame}
%
\begin{frame}[t,fragile] \frametitle{Auto-configuration}
	{\small
		\framesubtitle{Zero boilerplate code}
		\begin{itemize}[leftmargin=10pt,align=right]
			\onslide<2->\item[\alert{\faArrowCircleRight}] \alert{application.properties} definisce tutto
			\onslide<3->\item[\alert{\faArrowCircleRight}] \alert{@ConditionalOn*} annotations per configurazione condizionale
		\end{itemize}
		\vspace*{.3cm}
		\onslide<1->
\begin{minted}[bgcolor=vscode-dark-bg, bgcolorpadding=0.5em]{properties}
# OpenAI Configuration
spring.ai.openai.api-key=${OPENAI_API_KEY}
spring.ai.openai.chat.model=gpt-4
spring.ai.openai.chat.temperature=0.7

# Vector Store Configuration  
spring.ai.vectorstore.redis.uri=redis://localhost:6379
spring.ai.vectorstore.redis.index=spring-ai-index
spring.ai.vectorstore.redis.prefix=doc:

# Embedding Configuration
spring.ai.openai.embedding.model=text-embedding-ada-002
\end{minted}
	}
\end{frame}
%
\begin{frame}[t,fragile] \frametitle{Testabilità}
	{\small
		\framesubtitle{Testing semplificato}
		\onslide<1->
\begin{minted}[bgcolor=vscode-dark-bg, bgcolorpadding=0.5em]{java}
@SpringBootTest
class ChatServiceTest {
    
    @MockBean
    private ChatClient chatClient;
    
    @Autowired
    private ChatService chatService;
    
    @Test
    void shouldGenerateResponse() {
        // Given
        when(chatClient.prompt().user("Hello").call().content())
                .thenReturn("Hello! How can I help you?");
        
        // When  
        String response = chatService.chat("Hello");
        
        // Then
        assertThat(response).isEqualTo("Hello! How can I help you?");
    }
}
\end{minted}
	}
\end{frame}
%
\section{Svantaggi e Limitazioni} % (fold)
\label{sec:spring_ai_disadvantages}
%
\begin{frame}[t,fragile] \frametitle{Svantaggi di Spring AI}
	{\small
		\framesubtitle{Considerazioni e limitazioni}
		\begin{itemize}[leftmargin=10pt,align=right]
			\onslide<1->\item[\alert{\faArrowCircleRight}] \alert{Progetto giovane:} API ancora in evoluzione (versione 1.x)
			\onslide<2->\item[\alert{\faArrowCircleRight}] \alert{Astrazione overhead:} layer aggiuntivo rispetto alle API native
			\onslide<3->\item[\alert{\faArrowCircleRight}] \alert{Dipendenza dall'ecosistema Spring:} lock-in tecnologico
			\onslide<4->\item[\alert{\faArrowCircleRight}] \alert{Supporto provider limitato:} non tutti i provider sono supportati
			\onslide<5->\item[\alert{\faArrowCircleRight}] \alert{Funzionalità avanzate:} alcuni provider offrono feature specifiche non astratte
			\onslide<6->\item[\alert{\faArrowCircleRight}] \alert{Learning curve:} necessario apprendere nuovi concetti AI oltre Spring
		\end{itemize}
	}
\end{frame}
%
\begin{frame}[t,fragile] \frametitle{API in Evoluzione}
	{\small
		\framesubtitle{Rischi del progetto giovane}
		\begin{itemize}[leftmargin=10pt,align=right]
			\onslide<2->\item[\alert{\faArrowCircleRight}] \alert{Breaking changes:} possibili modifiche incompatibili tra versioni
			\onslide<3->\item[\alert{\faArrowCircleRight}] \alert{Documentazione incompleta:} esempi e best practices in sviluppo
			\onslide<4->\item[\alert{\faArrowCircleRight}] \alert{Community piccola:} meno risorse e troubleshooting disponibili
		\end{itemize}
		\vspace*{.3cm}
		\only<1-4>{
		\begin{center}
		\renewcommand{\epigraphsize}{\small}
		\setlength{\afterepigraphskip}{0pt}
		\setlength{\beforeepigraphskip}{5pt}
		\setlength{\epigraphwidth}{0.8\textwidth}
		\epigraph{\textit{Spring AI è attualmente in versione 1.0.1. Sebbene l'API principale sia stabile, alcune funzionalità potrebbero cambiare nelle future versioni minor.}}{Considerazione per progetti enterprise}
		\end{center}
		}
	}
\end{frame}
%
\begin{frame}[t,fragile] \frametitle{Overhead dell'Astrazione}
	{\small
		\framesubtitle{Il prezzo della portabilità}
		\onslide<1->
\begin{minted}[bgcolor=vscode-dark-bg, bgcolorpadding=0.5em]{java}
// Chiamata diretta OpenAI (senza Spring AI)
OpenAiService service = new OpenAiService("sk-...");
CompletionRequest request = CompletionRequest.builder()
    .model("gpt-4")
    .prompt("Explain Spring Boot")
    .temperature(0.7)
    .maxTokens(150)
    .build();
CompletionResult result = service.createCompletion(request);

// VS Spring AI (più semplice ma con overhead)  
ChatClient client = ChatClient.builder().build();
String response = client.prompt()
    .user("Explain Spring Boot")
    .call()
    .content();
\end{minted}
		}
		\only<2>{
		\begin{center}
			\alert{Semplicità vs Performance: trade-off da valutare}
		\end{center}
		}
\end{frame}